\message{ !name(letter_vAE.tex)}
\documentclass[apj]{emulateapj}

\usepackage{graphicx}
\usepackage{amssymb}
\usepackage{amsmath}
\usepackage{natbib}
\usepackage{color}
\bibliographystyle{apj}
\usepackage[breaklinks,colorlinks,citecolor=blue,linkcolor=magenta]{hyperref}
\shortauthors{Zhou et al.}

%%% new command %%%
\newcommand{\ima}{\texttt{ima} files }
\newcommand{\flt}{\texttt{flt} files }
\newcommand{\eps}{$\mathrm{e}^{-}/\mathrm{s}$}
\newcommand{\tinytim}{\textit{Tiny Tim}}
\newcommand{\bpic}{$\beta$ Pic}
\newcommand{\vsini}{$v\sin i$}
\newcommand{\mjup}{M$_{\mbox{Jup}}$}
\begin{document}

\message{ !name(letter_vAE.tex) !offset(-3) }


\title{Discovery of Rotational Modulations in the Planetary-mass
  Companion 2M1207b: A Slow Rotation Period and Cloud Height in a Low
  Gravity Atmosphere}
\shorttitle{Variability of 2M1207 b}
\author{Yifan Zhou\altaffilmark{1}, D\'aniel Apai\altaffilmark{1,2,3},
  Glenn Schneider\altaffilmark{1},  Mark S. Marley\altaffilmark{4}}

\altaffiltext{1}{Department of Astronomy/Steward Observatory, The
  University of Arizona, 933 N. Cherry Ave., Tucson, AZ, 85721, USA,
  \href{mailto:yifzhou@email.arizona.edu}{yifzhou@email.arizona.edu}
}
\altaffiltext{2}{Lunar and Planetary Laboratory, The University of
  Arizona, 1640 E. University Blvd., Tucson, AZ 85718, USA}
\altaffiltext{3}{Earths in Other Solar Systems Team, NASA Nexus for
  Exoplanet System Science}
\altaffiltext{4}{NASA Ames Research Center, Naval Air Station,
  Moffett Field,Mountain View, CA 94035, USA}

\begin{abstract}
  Rotational modulations of brown dwarfs have recently provided
  powerful constraints on the properties of ultra-cool atmospheres,
  including longitudinal and vertical cloud structures and cloud
  evolution. Furthermore, periodic light curve directly probes the rotational periods of ultra-cool objects.

  We present here, for the first time, time-resolved high-precision
  photometric measurements of a planetary-mass companion, 2MASS1207b,
  to a brown dwarf primary.  We observed the binary system with
  HST/WFC3 in two bands and with two spacecraft roll angles. Using
  point spread function-based photometry we reach a nearly
  photon-noise limited photometric accuracy for both components. While
  the primary is consistent with a flat lightcurve, the secondary
  shows modulations that are clearly detected in the combined
  lightcurve as well as in different subsets of the data.  The
  amplitudes are 1.45\% in the F125W and 0.92\% in the F160W filters;
  we find a consistent period of $10.2^{+0.9}_{-0.8}$ h and similar
  phase in both bands. {\color{red}The relative amplitudes in the two filters are
    very similar to that found in a recent study of a field
    (high-gravity) L-dwarf, suggesting that the cloud structures that
    introduce the photometric modulations are similar in high- and
    low-gravity objects.} Importantly, our study also measures, for
  the first time, the rotational period for directly an imaged
  planetary-mass companion.
\end{abstract}

\keywords{brown dwarfs -- planets and satellites: atmospheres -- planets
  and satellites: individual (2M1207 b) -- techniques: photometric}
\maketitle
%

\message{ !name(letter_vAE.tex) !offset(494) }

\end{document}

%%% Local Variables:
%%% mode: latex
%%% TeX-master: t
%%% End:
