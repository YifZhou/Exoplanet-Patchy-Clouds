\documentclass[apj]{emulateapj}

\usepackage{graphicx}
\usepackage{amssymb}
\usepackage{amsmath}
\usepackage{natbib}
\bibliographystyle{apj}
\usepackage[breaklinks,colorlinks,citecolor=blue,linkcolor=magenta]{hyperref} 
\shorttitle{Short Title}
\shortauthors{Zhou et al.}

%%% new command %%%
\newcommand{\ima}{\texttt{ima} files }
\newcommand{\flt}{\texttt{flt} files }
\newcommand{\eps}{$\mathrm{e}^{-}/\mathrm{s}$}
\begin{document}
\title{Variability of Planetary Mass Companion 2M1207 B}
\author{Yifan Zhou, Daniel Apai, ...}
\affil{University of Arizona}

\begin{abstract}
...
\end{abstract}

\keywords{kw1, kw2, ...}
\maketitle
%
\section{Introcduction}

\section{Observation}
We carried out high contrast direct imaging observation of  2M1207
system on UT 2014 April 11 using HST.
(\textit{HST} Proposal GO-13418, PI: D. Apai). The data were acquired
with the Wide Field Camera 3 (WFC3, citation needed) in filter F125W
and F160W using the $256\times256$ sub-array mode. We observed
2M1207 system in continuous 6 HST orbits and obtained a data-set that
have a temporal resolution of $\sim1.5$ minutes and baseline of $\sim
8$ hours.


The observation was initially designed to apply standard roll
subtraction to remove the primary star. The telescope roll angles for
data taken in orbit 1, 3, 5 and those taken in orbit 2, 4, 6 have a
difference of $25^{\circ}$, thus the companion would appear at
different position relative to the primary with different roll angle
and the primary in the two data set can be subtracted from each other.
In each orbit we will take a sequence of 13 SPARS25 exposures with
NSAMP=10, alternating between F160W and F125W filters. To improve PSF
sampling and reduce the influence of bad pixels, we applied standard 4
point dithering. In each orbit, exposures were taken with dithering
position 1 -- 4 sequently. To sum, we obtained 70 multi-acuum images
for filter F125W and 64 images for filter F160W with exposure time of
88.4 seconds for both two filters.


\section{Data Reduction}

Through the whole analysis, we used the \flt that were
produced by WFC3's \texttt{calwfc3} pipeline. The \flt were reduced
with calibrations including dark current correction, non-linearity
correction, and flat field correction. A further step of up-the-ramp
fit was applied to combine all non-destructive read and remove cosmic
rays. Although \citeauthor{Mandell2013} stated that WFC3 IR time series
extract from {\flt} have a rms 1.3 times larger than that obtained
from {\ima}, \flt keeps all pixels of the image of 2M1207 A
unsaturated while in most of non-destructive reads the cores of 2M1207
A are saturated, so that with \flt the flux of 2M1207 B can be more
precisely separated from that of 2M1207A. 

The small angular separation of 2M1207 A and B (as shown in Figure
\ref{fig:1}) makes precise primary star subtraction and photometry
very difficult. On the under-sampled WFC3 IR detector (plate scale
$\sim 0.13$ arcsec per pixel), the primary and the secondary only
separate by $\sim 6$ pixels, which is about 5 times of the FWHM of the
PSF. In addition, under-sampling of the PSF causes significant
artifacts when shifting the PSF to register images no matter what
interpolation method is used.

\begin{figure*}
  \centering
  \plottwo{original}{subtracted}
  \caption{WFC3 F160W images of 2M1207A system. Upper: original image,
    lower: residual model and primary PSF subtracted. The image of
    2M1207 B is overshadowed by the halo of the image of 2M1207A and
    can be hardly seen from the original image. With a hybrid PSF
    (residual + Tiny Tim PSF) subtraction, the image of 2M1207 B is
    clearly presented.}
  \label{fig:1}
\end{figure*}
\subsection{Tiny Tim PSF Photometry}
To over come the difficulty, we make use of the Tiny Tim PSF simulator
to pursue high precision photometry under this extreme
circumstance. Tiny Tim can produce model PSF based on the filter,
spectrum of target, focus status, and the telescope jitter. One
significant advantage of Tiny Tim PSF over empirical PSF is that Tiny Tim can
produce over-sampled PSF, which makes the shifting and interpolation
rather straight forward. However, Tiny Tim model PSFs have systematic
errors. \citeauthor{Biretta2014} demonstrated that the diffraction
spikes and coma are not well simulated with Tiny Tim for WFC3 IR images.

We start the reduction by making bad pixel masks for every
images. Pixels with data quality flags 'bad detector pixels' (DQ = 4),
'unstable response' (DQ = 32), and 'bad or uncertain flat value' (DQ =
512) were masked out and excluded from further analysis as suggested
by previous transit exoplanet
spectroscopic observations\citep[e.g.][]{Berta2012, Kreidberg2014}.

We then produced a list of 10x over-sampled PSFs based on the
positions and spectrum \citep{Bonnefoy2014} of 2M1207A with different telescope jitter
ranging from 0 to 50 mini-arcsec per second in both $x$ and $y$
direction. The focus parameter was calculated using the model listed
on the STScI
website\footnote{\url{http://www.stsci.edu/hst/observatory/focus/FocusModel}}. The
PSFs were registered with the original image by searching on a dynamic
grid and minimizing the residual of a region centered on 2M1207 A with
the image of 2M1207 B mostly excluded. The best fitted primary star
position and telescope jitter were obtained in this step. Then, we
generated the PSF for 2M1207 B based on its position on detector,
spectrum \citep{Patience2010}, and the telescope jitter that had been obtained above. In
the final step, we combined the two PSFs together. We fixed the
position of the primary as it had been fitted in the first step and
fitted for the amplitude of the two PSFs and the precise position of
the secondary. Since the total fluxes of the model PSFs are normalized
to unity as default, the fluxes of 2M1207 A and B are solely
represented by the amplitude of the two PSFs coordinately.

Because of the systematic errors of the Tiny Tim PSF, the quality of the
fitting is not optimistic. The reduced $\chi^{2}$ values are around
$\sim10$. However, we found that the residual patterns were very
stable among the image that have the same dithering position and
telescope rolling. Therefore, we median combined the residual images
that have the same dithering position and telescope rolling angle to
construct 8 residual models (4 dithering position $\times$ 2 telescope
rolling angle) for each filter. We pre-subtract the corresponding
residual from the original image and repeated the procedure that
listed above. Using this extra-step, the systematic error of Tiny Tim
greatly decreased to around unity.

The fluxes for both 2M1207 A and B are clearly correlated with the
dithering position, which is primarily due to imperfection of the flat
field. To reduce the correlation, we normalize each group of exposures
that have the same dithering position and roll angle indivadually --
we took the median of the fluxes that were measured from these
exposures as normalization factors and devided them from every flux
measurement. 


\section{Result}

We obtained high signal to noise photometry series for both 2M1207 A
and B. On average, the photometric contrast is $6.52\pm0.03$ for
F125W and $5.77\pm0.02$ for F160W. The difference of F125W contrast
from 7.0 magnitude of J-band contrast \citep{Mohanty2007} and that of
F160W contrast from 5.60 magnitude of NICMOS F160W contrast
\citep{Song2006} are due to the different throughput profiles of the
filters.

Normalized light curves for both filters are presented in Figure
\ref{fig:2}. Although the light curves are relative noisy and have a
scattering of $\sim 3\%$, both F125W and F160W light curves for 2M1207
B demonstrated a significant time variability.
\begin{figure*}[h]
  \centering
  \plotone{sineCurveFit}
  \caption{Normalized F125W and F160W light curves for 2M1207 B and
    A. For companion's light curves, gray crosses are unbinned
    photometric measurements while red circles and blue squares are
    exposure-set-based binned data representing two halves of the data
  set. For F160W, green curve is least square fitted sinusoidal wave
  with period fixed to be same as the fitting result for F125W, and
  purple curve is the one with no parameter fixed in the fitting routine.}
  \label{fig:2}
\end{figure*}

\subsection{Uncertainty analysis}
\begin{itemize}
\item the source of the scattering -- 
\item flat field uncertainty -- adding artificial flat field error mask
\item validity of the variability\\
  -- both filter demonstrate similar period of variability\\
  -- split the data into two half, the trend in two halves of data
  looks similar.
  -- fix the position does not change the light curve
\item uncertainty of sinusoidal curve fit
  -- carry out a mcmc fit? is it worth doing this.
\end{itemize}

\begin{figure*}[h]
  \centering
  \plottwo{F125_corr}{F160_corr}
  \caption{Fluxes of the primary and the secondary is correlated with
    the ditheirng positions. Different symbols represent different
    orbits.}
  \label{fig:3}
\end{figure*}

\section{Discussion }
\begin{itemize}
\item a data-reduction pipeline is developed to obtain high precision
  photometry measurement from high contrast WFC3 IR data. For a
  contrast of $\sim 7$ magnitudes at an angular separation of
  $\sim0.7''$, we obtained photometry measurement for 2M1207 B at
  precision of about 2-3\%. 
\item time variability for 2M1207 B was discovered, the light curve of
  2M1207 B for both two colors can be fitted with a T=10.7 hr
  sinusoidal curve.
\item constrain on the amplitude of the amplitude of the
  variability. Large amplitude can be excluded. Constraints on the
  inhomogeniety of the cloud coverage can be inferred -- small
  thickness variance...
\item ? the atmosphere and cloud structure of 2M1207 B. How does it
  compare to the brown dwarf.
\end{itemize}
\bibliography{ref.bib}
\end{document}

%%% Local Variables:
%%% mode: latex
%%% TeX-master: t
%%% End:
