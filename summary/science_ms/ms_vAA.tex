% Use only LaTeX2e, calling the article.cls class and 12-point type.

\documentclass[12pt]{article}

% Users of the {thebibliography} environment or BibTeX should use the
% scicite.sty package, downloadable from *Science* at
% www.sciencemag.org/about/authors/prep/TeX_help/ .
% This package should properly format in-text
% reference calls and reference-list numbers.

\usepackage{scicite}

% Use times if you have the font installed; otherwise, comment out the
% following line.

\usepackage{times}
\usepackage{hyperref}
% The preamble here sets up a lot of new/revised commands and
% environments.  It's annoying, but please do *not* try to strip these
% out into a separate .sty file (which could lead to the loss of some
% information when we convert the file to other formats).  Instead, keep
% them in the preamble of your main LaTeX source file.


% The following parameters seem to provide a reasonable page setup.

\topmargin 0.0cm
\oddsidemargin 0.2cm
\textwidth 16cm 
\textheight 21cm
\footskip 1.0cm


%The next command sets up an environment for the abstract to your paper.

\newenvironment{sciabstract}{%
\begin{quote} \bf}
{\end{quote}}


% If your reference list includes text notes as well as references,
% include the following line; otherwise, comment it out.

\renewcommand\refname{References and Notes}

% The following lines set up an environment for the last note in the
% reference list, which commonly includes acknowledgments of funding,
% help, etc.  It's intended for users of BibTeX or the {thebibliography}
% environment.  Users who are hand-coding their references at the end
% using a list environment such as {enumerate} can simply add another
% item at the end, and it will be numbered automatically.

\newcounter{lastnote}
\newenvironment{scilastnote}{%
\setcounter{lastnote}{\value{enumiv}}%
\addtocounter{lastnote}{+1}%
\begin{list}%
{\arabic{lastnote}.}
{\setlength{\leftmargin}{.22in}}
{\setlength{\labelsep}{.5em}}}
{\end{list}}

\newcommand{\ima}{\texttt{ima} files }
\newcommand{\flt}{\texttt{flt} files }
\newcommand{\eps}{$\mathrm{e}^{-}/\mathrm{s}$}
\newcommand{\tinytim}{\textit{Tiny Tim}}
\newcommand{\bpic}{$\beta$ Pic}
\newcommand{\vsini}{$v\sin i$}
\newcommand{\mjup}{M$_{\mbox{Jup}}$}
% Include your paper's title here

\title{First discovery of variability of direct imaged exoplanet 2M1207b} 


% Place the author information here.  Please hand-code the contact
% information and notecalls; do *not* use \footnote commands.  Let the
% author contact information appear immediately below the author names
% as shown.  We would also prefer that you don't change the type-size
% settings shown here.

\author
{Yifan Zhou,$^{1\ast}$ Daniel Apai,$^{1}$ Glenn Schneider$^{1}$ Mark
  Marley$^{2}$ Adam Showman$^{3}$\\
\\
\normalsize{$^{1}$Department of Astronomy, the University of Arizona}\\
\normalsize{933 N. Cherry Ave., Tucson 85721, USA}\\
\normalsize{$^{2}$NASA Ames Research Center}\\
\normalsize{$^{3}$Department of Planetary Science, the University of Arizona}\\
\\
\normalsize{$^\ast$To whom correspondence should be addressed; E-mail:
  yifzhou@email.arizona.edu.}
}

% Include the date command, but leave its argument blank.

\date{}



%%%%%%%%%%%%%%%%% END OF PREAMBLE %%%%%%%%%%%%%%%%



\begin{document} 

% Double-space the manuscript.

\baselineskip24pt

% Make the title.

\maketitle 



% Place your abstract within the special {sciabstract} environment.

\begin{sciabstract}
Rotational modulations in disk-integrated light of
brown dwarfs have recently provided powerful constraints on the
properties of ultracool atmospheres, including longitudinal and
vertical cloud structures and cloud evolution. Furthermore, detection
of periodic light curve variations can directly probe the rotational
periods of ultracool objects. We present here, for the first time, time-resolved high-precision
photometric measurements of a planetary-mass companion, 2MASS1207b, to
a brown dwarf primary. Using HST/WFC3 and point spread function %in
combination with two spacecraft roll angles we detect photometric
modulations in the light curve. The amplitude is 0.9\% in the F160W
and 1.5\% in the F125W filters; we find a consistent period and
similar phase in both bands. Joint fit to the lightcurve in both bands
suggest a period of $10.2^{+0.9}_{-0.8}$ h. The relative amplitudes in the two
filters are very similar to that found in a recent study of a field
(high-gravity) L-dwarf, suggesting that the cloud structures that
introduce the photometric modulations are similar in high- and
low-gravity objects. Importantly, our study also measures, for the
first time, the rotational period for directly imaged planetary-mass
companion.
\end{sciabstract}

\textbf{Introduction}

\textbf{Observation}We obtained direct images of the 2M1207A+b
system on UT 2014 April 11from 08:07:47 to 16:53:18 using the Hubble
Space Telescope (HST) and its Wide Field Camera 3 \cite[WFC3,
][]{Kimble2008} in the frame of the HST Proposal GO-13418 (PI:
D. Apai). We acquired the observations in filters F125W
($\lambda_{\mbox{pivot}}$ = 1245.9 nm, full width at half maximum
(FWHM) = 301.5 nm) and F160W ($\lambda_{\mbox{pivot}}$ 1540.52, FWHM =
287.9 nm), roughly corresponding to the J and H bands. The WFC3 pixel
scale is $\approx$13mas. We used the $256\times256$ pixels sub-array
mode to avoid memory dumps during the observations.  In order to
provide a near-continuous coverage for detecting modulations we
observed the 2M1207 system in six consecutive HST orbits, obtaining
data with cadence of $\sim1.5$ minutes over a baseline of 8 hours and
40 minutes. The observations were interrupted by 58 minute-long Earth
occultations every 94 minutes.

\textbf{Data reduction}
\textbf{Hybrid PSF photometry}

One major challenge of high contrast imaging observation using WFC3/IR
is significant under-sampling of the detector.  2M1207 A and b are
only separated by $\sim6$ pixels or $\sim$5 FWHM of the PSF. When
applying roll subtraction, notable artifact structures are generated
by image interpolation and shifting. On the other hand, \tinytim{} PSF
simulator\cite{Krist1995} offers a solution by providing Nyquist or better
sampled PSF, but systematic errors of \tinytim{} PSF for WFC3 limits
its ability in high precision photometry\cite{Biretta2014}. However,
we are able to fully characterize the difference of model and observed
PSFs with 6 orbits time-resolved observation data. To obtain robust \tinytim{} PSF
photometry, we designed a 2-round PSF fitting strategy: 1. calculating
correction map for \tinytim{}; 2. hybrid PSF photometry.

For both of 2 rounds, we used {\em Tiny Tim} to calculate 10$\times$
over-sampled model PSFs based on the filters, the spectra 
\cite{Bonnefoy2014, Patience2010}, the telescope's actual focus, and
the telescope jitter.  We used the new set of Tiny Tim parameters
provided by \cite{Biretta2014} to improve model of the cold mask,
OTA spikes, and the coma. The focus parameters were calculated
using the model listed on the STScI
website\footnote{\url{http://www.stsci.edu/hst/observatory/focus/FocusModel}}. To
register \tinytim{} PSF to observed PSF of 2M1207A, we moved the
over-sampled PSF over a coordinate grid (gird size=0.001 pixel) using
cubic interpolation and minimize the rms difference of the
observed and re-binned \tinytim{} PSFs over a region centered on 2M1207A with a
5-pixel-radius aperture centered on 2M1207b excluded. Then we introduce
another \tinytim{} PSF for 2M1207b and  fit the
position of 2M1207 b and the photometries of 2M1207 A and b simultaneously
by least square optimization. From our data set, we discovered that
the difference of observed PSFs and model PSFs were very stable for
given PSF positions. Therefore at the end of the first round PSF
fitting, we derived 8 (2 roll angles $\times$ 4 dithering positions)
observationally derived correction maps for each filter:
\begin{equation}
  \mathrm{Corr = Median(PSF_{obs.} - PSF_{model} )}
\end{equation}
where $\mathrm{PSF_{model}}$ is a combination of two \tinytim{} PSFs
for 2M1207 A and b. In the second round, we combined the correction
term linearly with the two \tinytim{} PSFs to generate hybrid PSFs,
and fitted the three components together. We found that by including the correction term,
the reduced $\chi^{2}$ of PSF fitting is decreased from $\sim 10$ to
around unity. Relative photometry is acquired from the scaling
parameters of the model PSFs.

PSF profiles change with exposure positions due to
pixelation, especially for the case that WFC3 IR is significantly
under-sampled. Also, the flat fields may potentially have large scale
structures \cite{dressel2012wide}. We found a correlation of
photometry with PSF positions on detector frame for both 2M1207 A and
b. Correction were made by normalizing each group of
exposures that have the same dithering position and roll angle
individually -- we took the median of the fluxes that were measured
from these exposures as normalization factors and divided them from
every photometric measurement. Because the normalization factor for
each group of exposures is calculated across the whole observation,
this normalization step have negligible impacts on variability
analysis.

\textbf{Result}
We present the first high-contrast, high-resolution, high-cadence, and
high-precision photometry of a directly imaged planet or
planetary-mass companion. Our observations reveal a modulation in the
light curve of the 5--7~M$_{J}$ companion 2M1207b, the first detection
of modulations in directly imaged ultracool objects.  The best fit
periods for F125W and F160W are 10.5 and 9.1 hour correspondingly. The
amplitudes for the normalized light curves are 1.45\% and 0.92\% for
F125W and F160W light curves.

The distributions for the periods demonstrate long tail shaped towards
long period, with core region roughly Gaussian. With 64\% confidence,
we estimated the 1-$\sigma$ range for the periods of F125W and F160W
to be $10.5_{-1.2}^{+1.3}$ and $9.1_{-1.0}^{+1.1}$ h,
respectively. The period of best fitted sine wave of F125W light
curve is 1.5h longer than that of F160W  that is slight larger
1-$\sigma$ standard deviation. We also jointly fit the two band light
curve together forcing the periods of two sine waves to be the
same. We derived a modulation period of $10.2^{+0.9}_{-0.8}$ h in this
circumstance. 

We discovered that  the variation amplitudes in the two bands were significantly
different. The distributions of the amplitudes are well described by
Gaussian profiles. By fitting a Gaussian function to the distribution, we
determined that amplitude distribution of F125W peaks at 1.45\% with a
standard deviation of 0.22\%, and that of F160W have mean and standard
deviation of 0.92\% and 0.20\%, respectively. The peaks of the two
histograms separated by more than 2-$\sigma$. The variation amplitude
of F125W light curve is 1.58 times of that of F160W light curve.

\textbf{Discussion}
A fundamental result of our study is the direct determination of the
rotation period of a directly imaged planetary-mass object. We
convert the rotation period to equatorial velocity  by adopting a radius of 1 -- 1.4
$R_{\mathrm{Jup}}$ for 2M1207b, and 1 $R_{\mathrm{Jup}}$ for field
brown dwarfs with well defined rotation period from the study of
\cite{Metchev2015}, and compare their rotation velocities with solar
system planets and \bpic{} b in the left
panel of Figure \ref{fig:5}.  The study by
\cite[][]{Snellen2014} succeeded in measuring \vsini{} for \bpic{} b
and demonstrated that it fits a trend defined by Solar System planets
in which more massive planets have faster rotation rates. They suggested
that this relation is linked to the accretion processes during planet formation.

Excitingly, our measurement of the rotation period demonstrates that
2M1207b, a planetary mass companion  with similar age to \bpic{} b, has a rotation velocity
that fits in the same trend, as well as majority of brown
dwarfs.
% % Periods are converted to equatorial rotation velocity by adopting a radius of 1 -- 1.4
% % $R_{\mathrm{Jup}}$ for 2M1207b, and 1 $R_{\mathrm{Jup}}$ for brown dwarfs with rotation period
% % measurement listed in \cite{Metchev2015}.
% In left panel of Figure \ref{fig:5}, we demonstrate that 2M1207b and
% most of the brown dwarfs falls on the same linear relationship with
% solar system planets and \bpic{} b on a
% $\log-\log$ plot of equatorial relationship vs. mass. 
We note that 2M1207b is most likely formed in the same way as brown
dwarfs by gravitational fragmentation, which suggests
that rotation periods are not good tracers of the formation pathways
and may not contribute important evidence for a formation in a disk
vs. in a cloud core environment.

Furthermore, our observations allow us to compare the relative
amplitudes in the J- and H-bands between the handful of brown dwarfs
for which high-quality near-infrared time-resolved observations have
been obtained. In the right panel of Figure \ref{fig:5}, we plot the
relative amplitude of J- and H-bands of 2M1207b with brown dwarfs
\cite{Apai2013,Buenzli2012,Buenzli2015,Burgasser2013,Radigan2012,Yang2014} that
have different spectral types and J$-$H colors.

We found a strong correlation between the spectral type of the object
and the J to H variation amplitude ratio. Figure \ref{fig:5}
demonstrates that earlier spectral type objects -- independent of
their surface gravity -- have larger variations at shorter wavelength
than at longer wavelengths. Importantly, although the J$-$H color of
2M1207b is significantly redder the other L5 dwarfs, we note that its
relative variation amplitude ratio is almost identical to the matching
spectral type mid-L dwarf 2M1821. 


% In setting up this template for *Science* papers, we've used both
% the \section* command and the \paragraph* command for topical
% divisions.  Which you use will of course depend on the type of paper
% you're writing.  Review Articles tend to have displayed headings, for
% which \section* is more appropriate; Research Articles, when they have
% formal topical divisions at all, tend to signal them with bold text
% that runs into the paragraph, for which \paragraph* is the right
% choice.  Either way, use the asterisk (*) modifier, as shown, to
% suppress numbering.

\textbf{Summary}In summary from our J- and H-band high precision,
high-cadence lightcurves we discovered sinusoidal modulations in the  planetary mass object
2M1207b. This is the first detection of rotational modulations in a
directly imaged planetary mass object.  
The period is  $10.2^{+0.9}_{-0.8}$, very similar to that derived from
{\em v sin i} measurements for  the direct imaged exoplanet \bpic{} b and significantly longer than
most field brown dwarfs with known rotation periods. The
relative modulation amplitude of J and H band is almost identical to
one matching spectral type L5 dwarf, although they have very different
J$-$H colors, and it is markedly different form later spectral type
brown dwarfs.

Finally, we note that the observations presented here open an exciting
new window on directly imaged exoplanets and planetary-mass
objects. Our study demonstrates a successful application of high-cadence,
high-precision, high-contrast photometry with planetary mass
companion. We also show that these observations can be carried out simultaneously
at multiple wavelengths, allowing us to prove multiple pressure
levels. With observation of a larger sample and at multiple wavelengths, we will be
able to explore  the detailed structures of atmospheres of directly
imaged exoplanets, and identify the key parameters that determine these.

\bibliography{ref.bib}

\bibliographystyle{Science}





\end{document}

%%% Local Variables:
%%% mode: latex
%%% TeX-master: t
%%% End:
