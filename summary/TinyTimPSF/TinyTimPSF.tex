%%%%%%%%%%%%%%%%%%%%%%%%%%%%%%%%%%%%%%%%%
% Short Sectioned Assignment
% LaTeX Template
% Version 1.0 (5/5/12)
%
% This template has been downloaded from:
% http://www.LaTeXTemplates.com
%
% Original author:
% Frits Wenneker (http://www.howtotex.com)
%
% License:
% CC BY-NC-SA 3.0 (http://creativecommons.org/licenses/by-nc-sa/3.0/)
%
%%%%%%%%%%%%%%%%%%%%%%%%%%%%%%%%%%%%%%%%%

%----------------------------------------------------------------------------------------
%   PACKAGES AND OTHER DOCUMENT CONFIGURATIONS
%----------------------------------------------------------------------------------------

\documentclass[paper=letter, fontsize=11pt]{scrartcl} % A4 paper and 11pt font size
\synctex=1
\usepackage[T1]{fontenc} % Use 8-bit encoding that has 256 glyphs
\usepackage{fourier} % Use the Adobe Utopia font for the document - comment this line to return to the LaTeX default
\usepackage[english]{babel} % English language/hyphenation
\usepackage{amsmath,amsfonts,amsthm} % Math packages
%\usepackage[nolists, nomarkers]{endfloat}
\usepackage{hyperref}
\usepackage{bm}
\usepackage{graphicx}
\usepackage[section]{placeins}
\usepackage{sectsty} % Allows customizing section commands
\allsectionsfont{\normalfont\scshape} % Make all sections centered, the default font and small caps

\usepackage{fancyhdr} % Custom headers and footers
\pagestyle{fancyplain} % Makes all pages in the document conform to the custom headers and footers
\fancyhead{} % No page header - if you want one, create it in the same way as the footers below
\fancyfoot[L]{} % Empty left footer
\fancyfoot[C]{} % Empty center footer
\fancyfoot[R]{\thepage} % Page numbering for right footer
\renewcommand{\headrulewidth}{0pt} % Remove header underlines
\renewcommand{\footrulewidth}{0pt} % Remove footer underlines
\setlength{\headheight}{13.6pt} % Customize the height of the header

\numberwithin{equation}{section} % Number equations within sections (i.e. 1.1, 1.2, 2.1, 2.2 instead of 1, 2, 3, 4)
\numberwithin{figure}{section} % Number figures within sections (i.e. 1.1, 1.2, 2.1, 2.2 instead of 1, 2, 3, 4)
\numberwithin{table}{section} % Number tables within sections (i.e. 1.1, 1.2, 2.1, 2.2 instead of 1, 2, 3, 4)

\setlength\parindent{0pt} % Removes all indentation from paragraphs -
                          % comment this line for an assignment with
                          % lots of text
\setlength\parskip{12pt}

%----------------------------------------------------------------------------------------
%   TITLE SECTION
%   ----------------------------------------------------------------------------------------
%----------------------------------------------------------------------------------------
%   TITLE SECTION
%----------------------------------------------------------------------------------------

\newcommand{\horrule}[1]{\rule{\linewidth}{#1}} % Create horizontal rule command with 1 argument of height
\newcommand{\chisq}{\ensuremath{\chi^2}}
\title{ 
\normalfont \normalsize 
\textsc{TinyTim PSF} \\ [25pt] % Your university, school and/or department name(s)
\horrule{0.5pt} \\[0.4cm] % Thin top horizontal rule
\huge Data Preparation\\ % The assignment title
\horrule{2pt} \\[0.5cm] % Thick bottom horizontal rule
}

\author{Yifan Zhou} % Your name

\date{\normalsize\today} % Today's date or a custom date

\begin{document}

\maketitle % Print the title

This document is a summary of using TinyTim PSF to measure the
photometry of 2M1207 system.

2M1207 system has a very small separation between the primary and
secondary. The peak of primary and secondary separates about 6
pixels. A perfect match of primary image and PSF is required at an
extreme level to ensure a good photometry measurement of
2M1207b. Image of the same system with different roll angle is an
option for PSF subtraction. However, considering coarse sample rate of
WFC3 IR, to shift the image can generate large artifact no matter what
interpolation method is using. Also, in our data, the angle telescope
rolled is ~30 degree, which could cause self-subtraction of the
secondary.

The advantage of TinyTim PSF is that TinyTim can produce 10x super
sampled PSF so that image shifting and interpolation is no longer a
problem. The downside of TinyTim is that it has several knobs, such as
jittering, secondary mirror shift to tune and even using the best
matched parameter set, TinyTim PSF are not good at modeling
diffraction spiders and coma due to intrinsic problems. 

\section{Generate PSF}

\begin{enumerate}
  \item specify spectrum for PSF using \texttt{tiny1}, and generate
    the input script for \texttt{tiny2}
  \item using python script \texttt{PSF\_generator.py} to modify the
    input script and run \texttt{tiny2} and \texttt{tiny3} to generate
    PSF\\
    Lines in the input script that need to be changed:
    
    \begin{itemize}
    \item line 2, the output file name
    \item line 8, the filter
    \item line10, major axis jitter in mas
    \item line 11, minor axis jitter in mas
    \item line 14, the x and y position on the detector, in integer
    \item line 260, secondary mirror displacement
    \end{itemize}
    \item the way to generate a 10x sampled PSF using \texttt{tiny3}:
      \texttt{tiny3 input.script sub=10}
\end{enumerate}

\section{Least \chisq Fit}

To match the center of PSF and star image, fit the PSF to image in a
grid of $dx$ and $dy$. The least \chisq fit calculates the amplitude
of PSF and a sky level. Amplitude and sky level can be solved
analytically.
\newcommand{\PSF}{\ensuremath{\mathrm{PSF}}}
\newcommand{\Img}{\ensuremath{\mathrm{Img}}}
\begin{equation}
  \begin{split}
    \chisq &= \sum_{i} \frac{(\mathrm{Img}_{i} - a\cdot\mathrm{PSF}_{i} -
      b)^{2}}{\sigma_{i}^{2}}\\
    &= \sum_{i} \frac{\mathrm{Img}_{i}^{2}+a^{2}\cdot \mathrm{PSF}_{i}^{2}+b^{2}+2ab\cdot \mathrm{PSF}_{i}
      -2a\cdot \mathrm{Img}_{i}\PSF_{i}-2b \mathrm{Img}_{i}}{\sigma_{i}^{2}}
  \end{split}
    \label{eq:chisq}
\end{equation}
To obtain least \chisq, $\frac{\partial \chisq}{\partial a} = 0$,
$\frac{\partial \chisq}{\partial b}=0$. So that
\begin{equation}
  \label{eq:calchisq}
  \begin{bmatrix}
    &\sum_{i}\frac{\PSF_{i}^{2}}{\sigma_{i}^{2}}
    &\sum_{i}\frac{\PSF_{i}}{\sigma_{i}^{2}}\\
    &\sum_{i}\frac{\PSF_{i}}{\sigma_{i}^{2}}
    &\sum_{i}\frac{b}{\sigma_{i}^{2}}
  \end{bmatrix}
  \cdot
  \begin{bmatrix}
    &a\\
    &b
  \end{bmatrix}
  =
  \begin{bmatrix}
    &\sum_{i}\frac{\Img_{i}\cdot\PSF_{i}}{\sigma_{i}^{2}}\\
    &\sum_{i}\frac{\Img_{i}}{\sigma_{i}^{2}}
  \end{bmatrix}
\end{equation}
$a$ and $b$ can be easily solved using simple linear algebra.

\section{Fit 2 PSFs together.}
Simply change equation \ref{eq:calchisq} to
\newcommand{\PSFI}{\ensuremath{\mathrm{PSF1}}}
\newcommand{\PSFII}{\ensuremath{\mathrm{PSF2}}}
\begin{equation}
  \label{eq:1}  
  \begin{bmatrix}
    & \sum_{i} \frac{\PSFI_i^2}{\sigma^2}
    &\sum_{i}\frac{\PSFI_i\cdot\PSFII_i}{\sigma_i^2}
    &\sum_{i}\frac{\PSFI_i}{\sigma_i^{2}}\\
    & \sum_{i} \frac{\PSFI_i\cdot\PSFII_{i}}{\sigma^2}
    &\sum_{i}\frac{\PSFII_i^{2}}{\sigma_i^2}
    &\sum_{i}\frac{\PSFII_i}{\sigma_i^{2}} \\
    &\sum_{i}\frac{\PSFI_{i} }{\sigma_{i}^{2}}
    &\sum_{i}\frac{\PSFII_{i} }{\sigma_{i}^{2}}
    &\sum_{i}\frac{b}{\sigma_{i}^{2}}
  \end{bmatrix}
  \cdot
  \begin{bmatrix}
    &a_1\\
    &a_2\\
    & b
  \end{bmatrix}
  =
  \begin{bmatrix}
    &\sum_{i}\frac{\Img_i\PSFI_i}{\sigma_i^2}\\
    &\sum_i \frac{\Img_i\PSFII_i}{\sigma_i^2}\\
    &\sum_{i}\frac{\Img_{i}}{\sigma_{i}^{2}}
  \end{bmatrix}
\end{equation}
\end{document}
%%% Local Variables:
%%% mode: latex
%%% TeX-master: t
%%% End:
